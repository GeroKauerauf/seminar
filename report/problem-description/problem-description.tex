\section{Problem description}
\label{sec:problem-description}

Since computers are getting better every year, one would expect that we could solve more and more problems.
This may be true in some fields, however, there are still problems that are very hard to solve.

Oftentimes, these problems are the ones that are hard to describe mathematically.
Take for example an image of a \emph{cat} and a \emph{dog}.
A human can intuitively recognize a cat on an image and is able to distinguish between those two animals with ease.
However, up to this day, it is not possible to describe the key characteristics of a \emph{cat} mathematically, 
in such a way, that one can build an algorithm that solves this problem.

Another class of problems is the one with problems that can be described mathematically perfectly but are too complex.
These are for example games like chess.
We have no problem with describing the rules of chess.
However, due to the sheer amount of possible moves and positions in chess, we have trouble analyzing games perfectly.
Writing an algorithm that \enquote{brute-forces} a game of chess is no problem, but in practice our computers are too slow.
Even worse, computers might never be fast enough!

This is where \emph{neural networks} come into play.
For these kinds of problems, they are currently the best solution we have.