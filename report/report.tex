% This is samplepaper.tex, a sample chapter demonstrating the
% LLNCS macro package for Springer Computer Science proceedings;
% Version 2.20 of 2017/10/04
%
\documentclass[runningheads]{llncs}
%
\makeatletter
\usepackage[algoruled,boxed,lined]{algorithm2e}
\makeatletter
\g@addto@macro{\@algocf@init}{\SetKwInOut{Parameter}{Parameters}}
\makeatother
\usepackage{amsmath}
\usepackage{amssymb}
\usepackage{graphicx}

% -------------------- My Packages -------------------- %
\usepackage{hyperref}
\usepackage{xcolor}
\usepackage[utf8]{inputenc}

%---------- Ich bin ein verdammter Künstler ----------
\usepackage{listings}
\definecolor{maroon}{RGB}{128, 0, 0}
\definecolor{pinegreen}{RGB}{1, 121, 111}
\definecolor{darkmidnightblue}{RGB}{0, 51, 102}
\definecolor{rwthblue}{RGB}{0, 84, 159}
% www.colorhexa.com for color references

% Quiet Light
\definecolor{background}{HTML}{f5f5f5}
\definecolor{keyword}{HTML}{4b83cd}
\definecolor{constant}{HTML}{ab6526}
\definecolor{function}{HTML}{aa3731} % bold
\definecolor{comment}{HTML}{aaaaaa} % italic
\definecolor{string}{HTML}{448c27}

\lstset{
    language=python,
    commentstyle=\color{comment}\textit,
    keywordstyle=\color{keyword},
    stringstyle=\color{string},
    frame=tb,
    keepspaces=true,
    stepnumber=2,
    showstringspaces=false,
    tabsize=2
}

% ---------- Hyperref -----------
\hypersetup{colorlinks,breaklinks,
            urlcolor=rwthblue,
            linkcolor=rwthblue}
\def\UrlBreaks{\do\/\do-}
% -------------------------------

% ------------------ End My Packages ------------------ %

% Used for displaying a sample figure. If possible, figure files should
% be included in EPS format.
%
% If you use the hyperref package, please uncomment the following line
% to display URLs in blue roman font according to Springer's eBook style:
% \renewcommand\UrlFont{\color{blue}\rmfamily}
% I don't really care.
\renewcommand\UrlFont{\rmfamily}


\begin{document}
% Load Python Syntax-Coloring
\lstloadlanguages{Python}
\lstset{
    basicstyle=\small,
    numbers=left,
    numberstyle=\footnotesize,
    stepnumber=1,
    numbersep=5pt,
    breaklines=true,
    escapeinside={/*@}{@*/}}

%
\title{Back-Propagation and Algorithms for Training Artificial Neural Networks with TensorFlow}
%
\titlerunning{Back-Propagation and Algorithms for Artificial Neural Networks}
% If the paper title is too long for the running head, you can set
% an abbreviated paper title here
%
\author{Gero Kauerauf\inst{1}}
%
\authorrunning{Kauerauf}
% First names are abbreviated in the running head.
% If there are more than two authors, 'et al.' is used.
%
\institute{RWTH Aachen University \email{} \href{https://github.com/gerokauerauf/seminar}{GitHub.com/GeroKauerauf/seminar}} 
%
\maketitle              % typeset the header of the contribution
%
\begin{abstract}
abstract
\keywords{keywords}
\end{abstract}
%
%
%
\section{Problem Description}

Sample citation: \cite{gelman2013bayesian}.

See Springer website\footnote{\url{https://www.springer.com/gp/computer-science/lncs/conference-proceedings-guidelines}.} for further inforamtion on the \LaTeX style.
\section{State of the Art}


\section{Contribution}

\section{Conclusion}

%
% ---- Bibliography ----
%
% BibTeX users should specify bibliography style 'splncs04'.
% References will then be sorted and formatted in the correct style.
%
\bibliographystyle{splncs04}
\bibliography{refs}

\newpage

\section{Testerino}
jaja kleiner Test

\lstinputlisting[language=Python]{python-code.py}

\end{document}
