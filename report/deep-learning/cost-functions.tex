\subsection{Cost Functions}
\label{sec:cost-functions}

A very important aspect of neural networks is the choice of a cost function.
In most cases we can make use of \emph{maximum likelihood} principle \cite{Cramer:444796}, from which we can than can derive a good cost function.
As thoroughly explained by \textbf{Goodfellow et al.} \cite{Goodfellow-et-al-2016}, we know the following: \\

Let \(p_{\text{model}}(\boldsymbol{x}; \boldsymbol{\theta})\) map any configuration \(\boldsymbol{x}\) to a real number that estimated the true probability \(p_{data}(\boldsymbol{x})\).
Then the maximum likelihood estimator for \(\boldsymbol{\theta}\) is defined by
\begin{align}
    \boldsymbol{\theta}_{ML} &= \argmax_{\boldsymbol{\theta}} p_{\text{model}} (\mathbb{X}; \boldsymbol{\theta}) \\
    &= \argmax_{\boldsymbol{\theta}} \prod^{m}_{i = 1} p_{\text{model}} (\boldsymbol{x}^{(i)}; \boldsymbol{\theta})
\end{align}
We can the take the logarithm of this product to obtain a convenient sum, that is less prone to numerical underflow.
\begin{equation}
    \boldsymbol{\theta}_{\textbf{ML}} = \argmax_{\boldsymbol{\theta}} \sum^{m}_{i = 1} \log p_{\text{model}} (\boldsymbol{x}^{(i)}; \boldsymbol{\theta})
\end{equation}
Dividing this by \(m\) yields
\begin{equation}
    \boldsymbol{\theta}_{\textbf{ML}} = \argmax_{\boldsymbol{\theta}} \mathbb{E}_{\fat{x} \sim \hat{p}_{\text{data}}} \log p_{\text{model}} (\boldsymbol{x}; \boldsymbol{\theta}).
\end{equation}

We can than generalize this to predict \(\boldsymbol{y}\) for a given \(\boldsymbol{x}\) by estimating a conditional probability \(P(\boldsymbol{Y} \vert \fat{x}; \boldsymbol{\theta})\).
Let \(\boldsymbol{X}\) be our inputs and \(\boldsymbol{Y}\) our observed outputs.
The conditional maximum likelihood estimator is
\begin{equation}
    \boldsymbol{\theta}_{\text{ML}} = \argmax_{\boldsymbol{\theta}} P (\boldsymbol{Y} \vert \boldsymbol{X}; \boldsymbol{\theta}).
\end{equation}
This further simplifies to
\begin{equation}
    \boldsymbol{\theta} = \argmax_{\boldsymbol{\theta}} \sum^{m}_{i = 1} \log P (\boldsymbol{y}^{(i)} \vert \boldsymbol{x}^{(i)}; \boldsymbol{\theta}),
\end{equation}
if we assume that the examples are independent and identically distributed .

This can then for example be used to show that the Mean-Squared-Error (MSE) works really well for linear regression.

