% This file provides an example Beamer presentation using the RWTH theme
% showcasing some of the more common options, similar to the Powerpoint version
% 12.11.2014: Revision 1 (Harold Bruintjes, Tim Lange)

% For RWTH, beamer should be loaded with class option t (top)
\documentclass[t]{beamer}

% Use fontspec to get Arial font
% Requires use of XeLaTeX
\usepackage{fontspec}
\setmainfont{Arial}
\setsansfont{Arial}
% Also force Arial for math for a more consistent look
\usepackage{unicode-math}

% https://tex.stackexchange.com/questions/426088/texlive-pretest-2018-beamer-and-subfig-collide
\makeatletter
\let\@@magyar@captionfix\relax
\makeatother

% German style date formatting (footer)
\usepackage[ddmmyyyy]{datetime}
\renewcommand{\dateseparator}{.}

\usepackage{MnSymbol,wasysym}

% Format the captions used for figures etc.
\usepackage[compatibility=false]{caption}
\captionsetup{singlelinecheck=off,justification=raggedleft,labelformat=empty,labelsep=none}

% PGFPlots is used for drawing some of the charts
\usepackage{pgfplots}
\pgfplotsset{compat=newest}
\input{plot_commands.tex}

% Load the actual RWTH theme. Suggested is to load the full theme,
% as it requires some specific dimensions
\usetheme{rwth}

% -------------------- My Packages -------------------- %
\usepackage{hyperref}
\usepackage{xcolor}
\usepackage[utf8]{inputenc}

% \usepackage{fontspec}
\setmonofont{Roboto Mono}
\usepackage{minted}

%---------- Ich bin ein verdammter Künstler ----------
\usepackage{listings}
\definecolor{maroon}{RGB}{128, 0, 0}
\definecolor{pinegreen}{RGB}{1, 121, 111}
\definecolor{darkmidnightblue}{RGB}{0, 51, 102}
\definecolor{rwthblue}{RGB}{0, 84, 159}
% www.colorhexa.com for color references

% ---------- Hyperref -----------
\hypersetup{colorlinks=true,
            breaklinks=true,
            urlcolor=rwthblue,
            linkcolor=rwthblue,
            citecolor=rwthblue}
\def\UrlBreaks{\do\/\do-}
% -------------------------------


\begin{document}
\logo{\includegraphics{logo.png}}

% Setup presentation information
\title{Back-Propagation and Algorithms for Training Artificial Neural Networks with TensorFlow}
\date{10. Dezember 2020}
\author{Gero Kauerauf}

\frame{\titlepage}

\section{Überblick}
% Frame with items
\begin{frame}
    \begin{itemize}
        \item Bildklassifizierung mit TensorFlow.Keras
        \begin{itemize}
            \item Was soll unser Modell können?
            \item Unser Datensatz
            \item Konstruktion unseres Python-Programms
        \end{itemize}
    \end{itemize}
\end{frame}

\section{Was soll unser Modell können?}
\begin{frame}
    \begin{itemize}
        \item Wir möchten ein Modell, welches unterschiedliche Blumen erkennen kann
        \item Wir haben eine Sammlung an von Menschen klassifizierten Bildern von Blumen
        \includegraphics[width=0.5\textwidth]{teach-plots/flower-photos}
        \item Enthalten sind Bilder von Gänseblümchen, Löwenzahn, Rosen, Sonnenblumen und Tuplen
        \item Mit diesem Datensatz können wir ein Modell bauen, welches zwischen genau diesen unterscheiden kann
    \end{itemize}
\end{frame}

\section{Der Datensatz}
\begin{frame}
    \begin{itemize}
        \item Hier ein kleiner Einblick in unseren Datensatz
    \end{itemize}
    \begin{figure}
        \centering
        \begin{minipage}{0.4\textwidth}
            \centering
            \includegraphics[width=0.8\textwidth]{./teach-plots/dandelion.jpg} % first figure itself
        \end{minipage}\hfill
        \begin{minipage}{0.4\textwidth}
            \centering
            \includegraphics[width=0.5\textwidth]{./teach-plots/rose.jpg} % second figure itself
        \end{minipage}
    \end{figure}
    \begin{figure}
        \centering
        \begin{minipage}{0.4\textwidth}
            \centering
            \includegraphics[width=0.8\textwidth]{./teach-plots/sunflower.jpg} % first figure itself
        \end{minipage}\hfill
        \begin{minipage}{0.4\textwidth}
            \centering
            \includegraphics[width=0.68\textwidth]{./teach-plots/tulip.jpg} % second figure itself
        \end{minipage}
    \end{figure}
\end{frame}

\section{Code}
\begin{frame}
    \begin{itemize}
        \item ...
    \end{itemize}
\end{frame}

\end{document}
